\documentclass[12pt]{article}


%%%%%%%%%%%%%%%%%%%%%%%%%%%%%%
%%%%%%%%Preamble%%%%%%%%%%%%%%
%%%%%%%%%%%%%%%%%%%%%%%%%%%%%%

%Packages ------------------------------------

\usepackage{amsmath}
\usepackage{listings}
\usepackage{lmodern}
\usepackage{amssymb,amsmath}
\usepackage{ifxetex,ifluatex}
\usepackage{lipsum}
\usepackage[T1]{fontenc}
\usepackage[utf8]{inputenc}
\usepackage[margin=1in]{geometry}
\addtolength{\skip\footins}{2pc plus 5pt} %Add whitespace before footnotes`
\usepackage[svgnames, x11names]{xcolor}
%\usepackage{tgadventor}

%Hyperlinks-----------------------------------
\usepackage{hyperref}
\hypersetup{
	colorlinks,
	citecolor=black,
	filecolor=black,
	linkcolor=black,
	urlcolor=black,
	 colorlinks=true, %set true if you want colored links
	linktoc=all     %set to all if you want both sections and subsections linked
}

%Remove Section Numbers----------------
\makeatletter
\renewcommand{\@seccntformat}[1]{}
\makeatother

%Change the Font------------------------------
%\usepackage{PoiretOne}
\renewcommand*\familydefault{\sfdefault} %% Only if the base font of the document is to be sans serif

% Add rule after sections ------------------------------
\usepackage{titlesec}






%Listings----------------------------------------


\definecolor{dkgreen}{rgb}{0,0.6,0}
\definecolor{gray}{rgb}{0.5,0.5,0.5}
\definecolor{mauve}{rgb}{0.58,0,0.82}
\lstset{
%  frame=tb,
  frame=leftline,
  framesep=15pt,
  language=Java,
  aboveskip=20pt,
  belowskip=20pt,
  showstringspaces=false,
  columns=flexible,
  basicstyle={\small\ttfamily},
  numbers=none,
%  backgroundcolor=\color{Snow2},
  numberstyle=\tiny\color{gray},
  keywordstyle=\color{blue},
  commentstyle=\color{dkgreen},
  stringstyle=\color{mauve},
  breaklines=true,
  breakatwhitespace=true,
  tabsize=3,
  xleftmargin=1in,
}

% Formatting----------------------------------------
\widowpenalty=1000
\clubpenalty=1000




\title{Computer Algebra Exercises}
\author{Ryan G}





%%%%%%%%%%%%%%%%%%%%%%%%%%%%%%
%%%%%%Document%%%%%%%%%%%%%%%%
%%%%%%%%%%%%%%%%%%%%%%% %%%%%%%
\begin{document}

\maketitle

\tableofcontents

\fontfamily{cmss}\selectfont

\section{(04) Lists}	
\subsection{Problem 4.10}
How many numbers of the form $3\cdot  n^5 +  11$ are prime for $n \in \left[ 1, 2000 \right]$?
\paragraph{Solution}
This would be a good problem to solve `top-down`, if you want to solve a problem top down, use the  defines symbol \verb|:=| \footnote{TB [2.3]-Equalities; P. 43} , this is analogous to creating a function or a method and means you can simply can that as a \verb|main|  method at the end.\\

So laying out or problem we might start with something that looks like this:

%\hfill\begin{minipage}{\dimexpr\textwidth-3cm}
\begin{lstlisting}[language = Mathematica]
(* Create the Select function *)
    returnvals := Select[vals897, PrimeQ]
        (* Put the code here*)
    returnvals
\end{lstlisting}	
%\end{minipage}

Now all that is necessary is to fill in the undefined methods/variables/functions:

\begin{itemize}
  \item \verb|vals897| will be a list of prime numbers which we could determine like so:
    \begin{lstlisting}[language = Mathematica]
         vals897 := Table[functionp410[i], {i, 0, maxval}] // Length
         \end{lstlisting}


         \begin{itemize}
           \item and from that we need to define the function $3n^5 +11$ and the maxval:
    %\hfill\begin{minipage}{\dimexpr\textwidth-3cm}
    \begin{lstlisting}[language = Mathematica]
        (* Define the Function and Domain *)
        functionp410[n_] := 3*n^5 + 11;
        maxval = 20
    \end{lstlisting}	
    %\end{minipage}

         \end{itemize}
\end{itemize}

So putting this all together we get:

%\hfill\begin{minipage}{\dimexpr\textwidth-3cm}
\begin{lstlisting}[language = Mathematica]
(* Create the Select function *)
returnvals :=  Select[vals897, PrimeQ] // Length

    (* Create a list of resulting values *)
    vals897 := Table[functionp410[i], {i, 0, maxval}];

        (*Define the Function and Domain*)
        functionp410[n_] := 3*n^5 + 11;
        maxval = 2000;

returnvals
\end{lstlisting}	
%\end{minipage}j

%\hfill\begin{minipage}{\dimexpr\textwidth-3cm}
\begin{lstlisting}[language = Mathematica]
98
\end{lstlisting}	
%\end{minipage}

And we hence conclude that there are 98 numbers of the form $3n^5+11, \quad \forall n \in \left[ 1, 2000 \right]$.




\begin{align*}
  4x^3&=6x\\
x&=\pm4x^3 
\end{align*}
\subsection{Exercise 4.1}	
Generate 6 random integers $n$ between $1$ and 3095 and show that $n^6 +  1091$ is not prime.

\paragraph{Solution}
In order to solve this first create a list of 6 random numbers using \verb|RandomInteger|:

\begin{lstlisting}[language = Mathematica]
Myvals = Table[RandomInteger[{1, 3095}], 6];
\end{lstlisting}	


Appending the \verb|;| character will cause the output will be supressed, now define the function:

\begin{lstlisting}[language = Mathematica]
f[n_] = n! + 1091;
\end{lstlisting}	

Now we can use \verb|/@| to apply the function over the list \footnote{TB [4.2]-Listable Functions P. 74} and then use \verb|//| to pipe that into the  \verb|PrimeQ| Function:

\begin{lstlisting}[language = Mathematica]
truthlist = 
 f /@ Myvals // PrimeQ (* 
  applying functions over a list is acheived by f[x] @ / {} or Map[f,
  expr], refer to p. 74 listable functions [4.2] *)
\end{lstlisting}	

This will return the output:

\begin{lstlisting}[language = Mathematica]
{False, False, False, False, False, False}
\end{lstlisting}	

But it isn't necessary to read that, instead we can execute:

\begin{lstlisting}[language = Mathematica]
If[AllTrue[f /@ Myvals, PrimeQ],
 	Print["These values are prime"],
 	Print["These values are composite"]
 ]
\end{lstlisting}	


\begin{lstlisting}[language = Mathematica]
These values are composite
\end{lstlisting}	

And we hence conclude that the random values are composite.

	
	



\subsection{Exercise 4.2}
Explain the Following Code:
%\hfill\begin{minipage}{\dimexpr\textwidth-3cm}
\begin{lstlisting}[language = Mathematica]
Text[Style["Sydney", Red, Italic, #]] & /@ Range[40, 70, 10]
\end{lstlisting}	
%\end{minipage}
\end{document}
