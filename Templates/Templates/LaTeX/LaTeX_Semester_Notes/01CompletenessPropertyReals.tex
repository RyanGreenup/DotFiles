\documentclass{article}
\usepackage{./resources/style}
\title{Completeness Property of the Reals}
\begin{document}
	\section{[2.3] Completeness Property of the Reals}
	\subsection{Upper and Lower Bounds [2.3.1]}
	\paragraph{Upper Bound}
	An upper bound is any value greater than or equal to all elements of a set, e.g. $u$ is an upper bound of $A$ if: \\

	\begin{equation}
	  \forall s \in S, \exists u \in \mathbb{R} : u \geq s
	  \label{upperdef}
	\end{equation}


	\paragraph{Lower Bound}
	A lower bound is any value less than or equal to all elements of a set, e.g. $w$ is a lower bound of $A$ if: \\

	\begin{equation}
	  \forall s \in S, \exists w \in \mathbb{R} : w \leq s
	  \label{lowerdef}
	\end{equation}


	\subsection{Supremea and Infima [2.3.2]}
	\paragraph{Supremum}
	The suprema of a set is the smallest upper bound value of some set.\\ 
	This value would be the maximum value of the set if the set had a maximum value.\\
	Let V be the set of all upper bound values, $u$ is a suprema iff:

	\begin{equation}
	  u \leq v, \forall v \in V 
	  \label{supdef}
	\end{equation}

	\paragraph{Infimum}
	The infimum of a set is the largest lower bound value of some set. \\
	This value would be the maximum value of the set if the set had a maximum value.\\
	Let T be the set of all upper bound values, $w$ is a suprema iff:

	\begin{equation}
	  w \leq t, \forall t \in T 
	  \label{infdef}
	\end{equation}

	
\end{document}

