\documentclass[class=article, crop=false]{standalone}

\usepackage{../resources/style}

\begin{document}
\section{Problem {\small Tutorial 1 Question 3 (iv)}}
Show that for real-valued functions $f(x)$ and $g(x)$:

\begin{figure}
	\centering
	\includegraphics[width=0.7\linewidth]{/home/ryan/Dropbox/Analysis/MD_Working_Directory/05_Continuity/pic}
	\caption{}
	\label{fig:pic}
\end{figure}



\begin{align}
  \text{min}(f,g) = \frac{1}{2} \left( f\left( x \right) + g\left( x \right) \right) - \frac{1}{2} \mid f\left( x \right) -g\left( x \right) \mid
\end{align}

\section{Solution}

The trick to proving this is:

\begin{align}
  \mid f\left( x \right) -g\left( x \right) \mid &= 
  \begin{cases}
    f\left( x \right) - g\left( x \right) , \enspace
    \left( f\left( x \right) - g\left( x \right) \right) \geq 0\\ 
    g\left( x \right) - f\left( x \right) , \enspace
    \left( f\left( x \right) - g\left( x \right) \right) < 0\\ 
  \end{cases}
  \label{def1} \\
  &= 
  \begin{cases}
    f\left( x \right) - g\left( x \right) , \enspace
    f\left( x \right)   \geq g\left( x \right)\\ 
    g\left( x \right) - f\left( x \right) , \enspace
    f\left( x \right)   < g\left( x \right)\\ 
  \end{cases}
  \label{def2} \\
  &= 
  \begin{cases}
    f\left( x \right) - g\left( x \right) , \enspace
    \min \left( f, g \right) = g\left( x \right) \\
    g\left( x \right) - f\left( x \right) , \enspace
    \min \left( f, g \right) = f\left( x \right)
  \end{cases}
  \label{def3}
\end{align}

Now flowing from the \textit{Trichotomoy Princple} one of the following conditions must hold:

\begin{equation}
\min{\left( f, g \right)} = g \quad \vee \quad  \min{\left( f,g \right)} = f \quad \vee \quad f\left( x \right) = g\left( x \right)  \\
  \label{trichotomy}
\end{equation}

\subsection{Assume $f\left( x \right) = g\left( x \right)$}
\begin{align}
\frac{1}{2} \left( f\left( x \right) + g\left( x \right) \right) - \frac{1}{2} \mid f\left( x \right) -g\left( x \right) \mid &= 
\frac{1}{2} \left( f\left( x \right) + f\left( x \right) \right) - \frac{1}{2} \mid f\left( x \right) -f\left( x \right) \mid
\notag \\
&= \frac{1}{2} \left( f\left( x \right) + f\left( x \right) \right) \notag \\
&= \frac{1}{2} \left( 2 \times f\left( x \right)  \right) \notag \\
&=  f\left( x \right) = g\left( x \right) 
\end{align}

Given that it was assumed $f\left( x \right) = g\left( x \right)$ this result is consistent with this being the minimum value of the two functions.

\subsection{assume $\min{\left( f, g \right)} = g \left( x \right)$}

\begin{align}
\frac{1}{2} \left( f\left( x \right) + g\left( x \right) \right) - \frac{1}{2} \mid f\left( x \right) -g\left( x \right) \mid
&= \frac{1}{2} \left( f\left( x \right) + f\left( x \right) \right) - \frac{1}{2} \left( f\left( x \right) - g\left( x \right) \right) \notag \\
&= \frac{f\left( x \right)}{2} + \frac{g\left( x \right)}{2}  - \frac{f\left( x \right)}{2} +   \frac{g\left( x \right)}{2} \notag \\
&= \frac{g\left( x \right)}{2} +   \frac{g\left( x \right)}{2} \notag \\
&= g\left( x \right) \notag \\
&= \min{\left( f, g \right)} \dots \text{(From the initial assumption}
\end{align}

\subsection{assume $\min{\left( f, g \right)} = f \left( x \right)$}

\begin{align}
\frac{1}{2} \left( f\left( x \right) + g\left( x \right) \right) - \frac{1}{2} \mid f\left( x \right) -g\left( x \right) \mid
&= \frac{1}{2} \left( f\left( x \right) + g\left( x \right) \right) - \frac{1}{2} \left( g\left( x \right) - f\left( x \right) \right) \notag \\
&= \frac{f\left( x \right)}{2} + \frac{g\left( x \right)}{2} - \frac{g\left( x \right)}{2}  +   \frac{f\left( x \right)}{2} \notag  \\
&= \frac{f\left( x \right)}{2} = \frac{f\left( x \right)}{2} \notag \\
&= f\left( x \right) \notag \\
&= \min{\left( f, g \right)} \dots \text{From the initial assumption}
\end{align}

\section{Conclusion}
Given that the \textit{trichotomy principle} provides that the 3 above situations must hold and that in all cases the formula provides the minimum value, it stands that:

\begin{align}
  \text{min}(f,g) = \frac{1}{2} \left( f\left( x \right) + g\left( x \right) \right) - \frac{1}{2} \mid f\left( x \right) -g\left( x \right) \mid \notag
\end{align}

\end{document}
